\documentclass{article}
\usepackage[utf8]{inputenc}
\usepackage{graphicx,float}
\usepackage{hyperref}
\usepackage{listings}
\graphicspath{{imagenes/}}
\title{Programación de Servicios y Procesos: Práctica de Docker Local}
\author{Álvaro Del Valle Fernández}

\begin{document}

\maketitle

\section{Introducción}
El objetivo de esta práctica era crear una base de datos local, en este documento mostraré como cree los contenedores,
 la API y la base de datos, usando Docker, Docker Hub,GitHub y notificaciones en Discord. \\
 Tambien incluí un elemento .java para realizar pruebas y comprobar el correcto funcionamiento del sistema.

 \section{Requisitos Funcionales}
\subsection{Comunicación entre Servicios}
\begin{itemize}
    \item \textbf{RF-01}: Debe conectarse de forma automatica una vez iniciada
    \item \textbf{RF-02}: Debe notificar de posibles errores
    \item \textbf{RF-03}: Comunicacion mediante Docker
\end{itemize}

\section{Requisitos No Funcionales}
\begin{itemize}
    \item \textbf{RNF-01}: Debe de funcionar de forma fluida y sin esperas demasiado largas.
    \item \textbf{RNF-02}: Debe de estar organizado de forma modular permitiendo mejoras de forma sencilla.
\end{itemize}

\section{Arquitectura}
El sistema se compone de los siguientes elementos: 
\begin{itemize}
    \item \textbf{Backend}: Api con Node.js y Express
    \item \textbf{Contenedorización}: Docker
    \item \textbf{Registro}: Docker Hub
    \item \textbf{CI/CD}: GitHub
    \item \textbf{Notificaciones}: Discord con Webhooks
\end{itemize}
\subsection{Flujo de Datos}
\begin{enumerate}
    \item Al iniciar se conecta al puerto adecuado automaticamente.
    \item La API gestiona la petición
    \item La API se conecta a la base de datos mendiante Docker
    \item Devuelve los resultados
    \item La API devuelve la respuesta al usuario mediante Discord
\end{enumerate}
\section{Clases Principales}
\subsection{docker-push.yml}
\begin{lstlisting}
   name: Build and Push Docker Image

on:
  push:
    branches:
      - main
      - master
    paths:
      - 'api/**'
      - '.github/workflows/docker-push.yml'
  workflow_dispatch: # Permite ejecutar manualmente

env:
  DOCKER_IMAGE_NAME: ${{ secrets.DOCKER_USERNAME }}/ejemplo-api-docker
  DOCKER_TAG: latest

jobs:
  build-and-push:
    runs-on: ubuntu-latest
    
    steps:
    - name: Checkout codigo
      uses: actions/checkout@v4
      
    - name: Configurar Docker Buildx
      uses: docker/setup-buildx-action@v3
      
    - name: Login a Docker Hub
      uses: docker/login-action@v3
      with:
        username: ${{ secrets.DOCKER_USERNAME }}
        password: ${{ secrets.DOCKER_PASSWORD }}
        
    - name: Construir y subir imagen Docker
      uses: docker/build-push-action@v5
      with:
        context: ./api
        file: ./api/Dockerfile
        push: true
        tags: |
          ${{ env.DOCKER_IMAGE_NAME }}:${{ env.DOCKER_TAG }}
          ${{ env.DOCKER_IMAGE_NAME }}:v1.0.0
        cache-from: type=registry,ref=${{ env.DOCKER_IMAGE_NAME }}:buildcache
        cache-to: type=inline
        
    - name: Notify Discord - SUCCESS
      if: success()
      uses: sarisia/actions-status-discord@v1
      with:
        webhook: ${{ secrets.DISCORD_WEBHOOK }}
        title: "DEPLOY EXITOSO - Docker Image"
        description: |
          **Imagen construida y subida correctamente**
          
           **Imagen:** `${{ env.DOCKER_IMAGE_NAME }}:${{ env.DOCKER_TAG }}`
           **Tag:** v1.0.0
           **Estado:** Success
           **URL:** https://hub.docker.com/r/${{ secrets.DOCKER_USERNAME }}/api-simple
        username: "Docker Deploy Bot"
        avatar_url: "https://cdn-icons-png.flaticon.com/512/919/919853.png"
        
    - name: Notify Discord - FAILURE
      if: failure()
      uses: sarisia/actions-status-discord@v1
      with:
        webhook: ${{ secrets.DISCORD_WEBHOOK }}
        title: "DEPLOY FALLIDO - Docker Image"
        description: |
          **Fallo al construir o subir la imagen Docker**
          
           **Imagen:** `${{ env.DOCKER_IMAGE_NAME }}:${{ env.DOCKER_TAG }}`
           **Estado:** Failed
           **Workflow:** ${{ github.workflow }}
          
          *Revisa los logs de GitHub Actions*
        username: "Docker Deploy Bot"
\end{lstlisting}

\subsection{Archivo Dockerfile}
\begin{lstlisting}
    FROM node:18-alpine
WORKDIR /app
COPY package*.json ./
RUN npm install
COPY . .
EXPOSE 3000
CMD ["npm", "start"]
\end{lstlisting}

\subsection{Archivo package.json}
\begin{lstlisting}
   {
  "name": "ejemplo-api",
  "version": "1.0.0",
  "main": "server.js",
  "scripts": {
    "start": "node server.js",
    "dev": "nodemon server.js"
  },
  "dependencies": {
    "express": "^4.18.2",
    "pg": "^8.11.3",
    "dotenv": "^16.3.1",
    "cors": "^2.8.5",
    "express-validator": "^7.0.1"
  },
  "devDependencies": {
    "nodemon": "^3.0.1"
  }
}
\end{lstlisting}

\subsection{Archivo server.js}
\begin{lstlisting}
    const express = require('express');
  const app = express();
  app.use(express.json());

let usuarios = [
  { id: 1, username: 'admin', password: '123' },
  { id: 2, username: 'user1', password: '456' }
];
//webhook 
//rutas http://localhost:3000/
app.get('/', (req, res) => {
  res.json({
    mensaje: 'API funcionando',
    endpoints: [
      'GET /users',
      'POST /users',
      'POST /login'
    ]
  });
});

//GET USUARIO
app.get('/users', (req, res) => {
  res.json({
    usuarios: usuarios.map(u => ({ id: u.id, username: u.username }))
  });
});

//CREAR USUARIO
app.post('/users', (req, res) => {
  const { username, password } = req.body;
  
  if (!username || !password) {
    return res.status(400).json({ error: 'Faltan datos' });
  }
  
  const nuevoId = usuarios.length > 0 ? Math.max(...usuarios.map(u => u.id)) + 1 : 1;
  const nuevoUsuario = { id: nuevoId, username, password };
  
  usuarios.push(nuevoUsuario);
  
  res.status(201).json({
    mensaje: 'Usuario creado',
    usuario: { id: nuevoId, username }
  });
});

app.post('/login', (req, res) => {
  const { username, password } = req.body;
  
  const usuario = usuarios.find(u => 
    u.username === username && u.password === password
  );
  
  if (usuario) {
    res.json({
      mensaje: 'Login correcto',
      usuario: { id: usuario.id, username: usuario.username }
    });
  } else {
    res.status(401).json({ error: 'Credenciales incorrectas' });
  }
});

const PORT = process.env.PORT || 3000;
app.listen(PORT, () => {
  console.log(`Servidor en http://localhost:${PORT}`);
});
\end{lstlisting}
\section{Comandos Utilizados}
Para realizar esta practica utilize numerosos comandos, entre ellos: \\
Lanzar el server:\\
 npm installnode server.js\\
Crear y correr el contenedor:\\
 docker build -t prueba ./api \\
 docker run -p 3000:3000 prueba \\
 docker-compose up\\
 docker-compose down\\
\section{Capturas de Pantalla}
Test de comprobacion del funcionamiento de la API:
 \begin{figure}[H]
    \centering
    \includegraphics[width=4in]{imag1.png}
\end{figure}
comprobacion de los secrets de GitHub:
\begin{figure}[H]
    \centering
    \includegraphics[width=4in]{imag2.png}
\end{figure}
Server corriendo localmente:
 \begin{figure}[H]
    \centering
    \includegraphics[width=4in]{imag3.png}
\end{figure}
Notificación de comunicación correcta en Discord:
 \begin{figure}[H]
    \centering
    \includegraphics[width=4in]{discord.png}
\end{figure}
\section{Tecnologías Usadas}
\begin{itemize}
    \item \textbf{Node} - Entorno de ejecucion JS
    \item \textbf{Express} - Framework
    \item \textbf{Docker} - Para crear contenedores
    \item \textbf{Docker Hub} - Registro de imágenes Docker
    \item \textbf{GitHub} - Para automatización CI/CD
    \item \textbf{Discord Webhooks} - Notificaciones
\end{itemize}
\end{document}